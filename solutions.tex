\documentclass{article}
\usepackage{amsfonts} 
\usepackage{amsmath}
\usepackage{amssymb}
\usepackage{dcolumn}
\usepackage{tikz-cd}
\usepackage{bbold}

\newcolumntype{2}{D{.}{}{4.0}}
\title{Solutions: Probability Theory by S.R.S Varadhan}
\author{Hassaan Naeem}
\date{\today}
\begin{document}
\maketitle

\section*{Chapter 1. Measure Theory}
\subsection*{Exercise 1.11}
There is a difference between almost everywhere convergence...
\\\\
\textbf{Solution:}
We let $I_{2n} = \left (\frac{1}{2n}, 1 \right ]$ and $I_{2n+1} = \left [ 0, \frac{1}{2n+1} \right ]$.
We also let $f_n(x) = \mathbb{1}_{I_n}(x)$.
Then we have that $\lim_{n\to\infty} P[\omega:|f_n(\omega)-f(\omega)| \ge \epsilon] = 0, \ \forall \epsilon >0$.

\subsection*{Exercise 1.12}
But the following statement is true...
\\\\
\textbf{Solution:}

\end{document}
